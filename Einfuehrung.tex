%!TEX root = Junghackerfibel_Buch.tex
%!TEX encoding = UTF-8 Unicode

\chapter{Einführung}

\section{Was ist Hacken?}

Hacken ist eine Denkweise und eine Lebenseinstellung. Denkfaule Menschen fragen: Wofür muß ich das wissen? Hacker fragen: Wofür könnte ich das wohl verwenden? Hacken ist eine Denkweise, die das gesamte Leben durchzieht. Die Denkweise des Hackens läßt sich ebenso auf Kochen wie auf Sprache oder soziale Interaktion anwenden. 

Die ersten Hacker waren wohl diejenigen Menschen, die zwei Steine aufeinandergeschlagen und festgestellt haben, daß dabei scharfkantige Teile abplatzen können, oder wie man mit einem Stein und einem Stock Feuer machen kann. Hacker erblicken weitere wichtige Ahnen in beispielsweise dem Komponisten Johann Sebastian Bach, dem Dichter William Shakespeare oder dem Aufklärer und Verleger Joachim Heinrich Campe, der die deutsche Sprache vielleicht in ähnlich gewaltiger Weise bereichert hat wie Shakespeare die englische. 

Hacken ist also keineswegs auf Rechner und Netzwerke beschränkt, nicht einmal auf Technik, sondern bezieht sich auf Systeme aller Art. 

Dieses Buch befaßt sich mit einem umfassenden Einstieg ins Hacken, denn es ist festzustellen, daß viele Hacker \emph{nicht} damit angefangen haben, an Computern und Programmen herumzubasteln. Für viele war der Einstieg ein ganz anderer: Sprache. Für wieder andere begann es mit Legosteinen, ging über Holzbearbeitung zu Metallbearbeitung zu Elektronik. Beide Beispiele haben eines gemeinsam: eine niedrige Einstiegsschwelle. Der Ansatz dieses Buches will den Einstieg möglichst niedrig gestalten. Alles andere ergibt sich dann von alleine.

Wenn ein junger Mensch fragt: \emph{Wie werde ich Hacker?} werden viele ihn auf bestimmte Texte verweisen, welche die Hürde so hoch legen, daß sie als Einstieg illusorisch ist. Junge Menschen vom Hacken abzuhalten ist arrogant und widerspricht dem aufklärerischen und philantropischen Anspruch, den ein Hacker eigentlich haben sollte. Ich werde deshalb in diesem Buch auch immer wieder auf die ethischen Aspekte des Hackens eingehen. 