%!TEX root = Junghackerfibel_Buch.tex
%!TEX encoding = UTF-8 Unicode

\chapter{Lies und finde Spaß daran}

Gewöhne Dir an, regelmäßig zu lesen, zum Beispiel eine halbe Stunde vor dem Einschlafen oder eine halbe Stunde nach dem Erwachen – vielleicht zumindest am Wochenende. Je mehr Du liest, desto leichter wird es Dir fallen. Laß Dich nicht von der Dicke von Büchern abschrecken: Lies Zeile für Zeile, Seite für Seite – und irgendwann wirst Du feststellen, daß Du ein ganzes Buch gelesen hast. 

Frag andere Leute, welche Bücher sie gelesen haben und empfehlen können, Deine Eltern, Deine Freunde, Deine Lehrer. Es gibt Bücher, die sind der helle Wahnsinn. Finde diese Bücher! Menschen in Deinem Alter lesen zum Beispiel gerne \emph{Die drei ???}, \emph{Asterix}, \emph{Perry Rhodan}, Karl May oder auch die \emph{Lustigen Taschenbücher} – und natürlich \emph{Harry Potter}: \emph{Harry Potter} ist eine besondere Empfehlung, denn es ist vordergründig eine spannende Geschichte, aber sie ist doppelbödig, vielschichtig und tiefsinnig, also genau das, was gute Literatur ausmacht.

Frag, ob Du die Pause in der Schulbibliothek verbringen darfst: Wenn sowieso eine Aufsichtsperson da ist, gibt es keinen Grund, weswegen die Schule Dir das verweigern sollte. Vielleicht kannst Du Deine Freunde davon überzeugen, die Schulbibliothek als Euer Reich zu übernehmen. Du willst in der Pause lieber lernen als die Zeit auf dem Pausenhof zu vertrödeln, und die Schule hindert Dich daran? Das ist ja unglaublich! Bitte die Schülervertretung oder den Elternbeirat um Unterstützung.

In der Schulbibliothek könntest Du Dir zuerst einmal das Ordnungssystem, die Signaturen auf dem Buchrücken und den Katalog erklären lassen – oder Dir selbst erklären. Es ist völlig egal, was Du liest: Du kannst einen Asterix nach dem anderen lesen oder den Brockhaus in einem Band von A bis Z durchlesen oder mit geschlossenen Augen zufällig einen Artikel auswählen. Friß Dir Wissen an, je mehr Wissen Du Dir angefressen hast, desto leichter wirst Du Dir weitere Sachen merken können. Geh danach, was Dich interessiert: Wenn es Fußball ist, dann lies alles über Fußball!

Besorg Dir das Buch \emph{Bildung} von Dietrich Schwanitz – damit wirst Du es in allen nicht-naturwissenschaftlichen Fächern leichter haben, und Du hast mehr Zeit für Deine Hacksportaktivitäten.