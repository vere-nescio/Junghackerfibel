%!TEX root = Junghackerfibel_Buch.tex
%!TEX encoding = UTF-8 Unicode

\chapter{Voraussetzungen}

Dieses Buch ist geschrieben für Kinder ab 12 Jahren. Jugendliche und Erwachsene werden es aber ganz bestimmt genauso mit Gewinn lesen. 

Die Anleitungen und Beispiele in diesem Buch beziehen sich auf den Apple Macintosh, und zwar aus mehreren Gründen:

\begin{compactitem}[–]
\item Macintosh-Rechner sind ideale Rechner für Kinder, weil man sie ziemlich angstfrei verwenden kann, man sich keine größeren Sorgen um Viren und andere Schadprogramme machen muß, und sie eine niedrige Einstiegsschwelle in den Umgang mit Rechnern generell bieten.
\item Macintosh-Rechner sind in der Hacker-Szene mit Abstand die beliebtesten Geräte, denn man kann damit alles machen: Man hat Zugriff auf Macintosh-Programme, wenn man auch mal etwas Normales arbeiten muß, mehr oder weniger sämtliche Unix-\emph{Software}, kann mehr oder weniger sämtliche Betriebssysteme in einer Box laufen lassen und bekommt keine TPM-Komponente aufs Auge gedrückt, die man nicht haben will (TPM = \emph{Trusted Platform Module}), und das Ganze sieht überhaupt nicht zum Kotzen aus. Abgesehen davon sind die Geräte in der Regel schnell und langlebig, dafür verlangt Apple Premium-Preise, welche die Leute, die damit arbeiten, offenbar für ihr Arbeitsgerät zu zahlen bereit sind. Ein Schreiner kauft sich für sein Handwerker auch die beste Säge und den besten Hobel, und Kopfarbeiter sollten auch das beste Werkzeug wählen.
\item Menschen, die mit dem Macintosh großgeworden sind, kommen mehr oder weniger mit jedem anderen Rechner zurecht, wohingegen Leute, die mit Linux oder Windows großgeworden sind, nie lernen, daß Dinge auch einfach sein können. Der Sinn für Ästhetik und Einfachheit – nicht zuletzt die Ästhetik der Einfachheit – wird man mit einem Macintosh als Arbeitsgerät vermutlich so verinnerlichen, daß sich eine Aversion gegen unnötige, lieblose und denkfaule Kompliziertheit entwickelt – und das ist gut!
\end{compactitem}

\noindent Man kann also vom blutigen Anfänger bis zum Guru mit demselben Gerät arbeiten. Ich will hier weder Werbung machen noch große Diskussionen vom Zaun brechen. Deshalb: Genug davon!

Mit Windows kommt man als Hacker nicht weit, mehr gibt es dazu nicht zu sagen. Wer sein Windows vom Rechner zu werfen bereit ist oder vor der Anschaffung eines Rechners steht, möge sich Linux – Ubuntu, Kubuntu, Mint – oder gleich PC-BSD anschauen. Irgendwie wird es auch möglich sein, Windows und Linux bzw. PC-BSD zu einer friedlichen Koexistenz auf demselben Rechner überreden lassen, aber ich werde nie erfahren, wie das geht. Viele der Kommandozeilenbefehle werden unter Unix-artigen Betriebssystemen genauso oder ähnlich funktionieren.

Es muß nicht der neueste Macintosh sein. Für die jüngsten Leser reicht ein alter Macintosh, er sollte aber bereits einen Intel-Prozessor besitzen, und es sollte zumindest Mac OS X 10.6 \emph{Snow Leopard} darauf laufen: So ein Gerät bekommt man mitunter geschenkt oder günstig im Internet. Wenn Du größer wirst, wirst Du irgendwie schon einen neueren Macintosh bekommen, und der alte Macintosh wird dann Dein Bastelgerät, auf dem Du FreeBSD installierst und Deinen ersten eigenen Server aufsetzt.