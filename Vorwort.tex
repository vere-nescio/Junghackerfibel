%!TEX root = Junghackerfibel_Buch.tex
%!TEX encoding = UTF-8 Unicode

\chapter{Vorwort}

Dieses Buch will den Einstieg ins Hacken erleichtern. Während andere Einführungen die Hürden möglichst hoch legen, will dieses Buch sie möglichst niedrig legen und zugleich Fehlentwicklungen möglichst frühzeitig entgegenwirken. Wir wollen hier deshalb weder gleich am Anfang den Umstieg auf ein Unix-artiges Betriebssystem empfehlen noch gleich mit Programmierung einsteigen, sondern wir wollen zuerst die Denkweise vermitteln, danach folgt alles weitere – wahrscheinlich sogar mehr oder weniger von selbst. Früher oder später landen die meisten bei Unix-artigen Betriebssystemen und Programmierung.

Der Weg zum Hacken kann ganz unterschiedlich sein. Bei manchen ist der Einstieg ein spielerischer Umgang mit Sprache, andere fangen an mit Malen und Zeichnen, landen dann bei Computer-Graphik und Satzgestaltung und lernen dann nach und nach ihren Computer besser kennen.  