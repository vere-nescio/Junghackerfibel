% !TEX encoding = UTF-8 Unicode

\documentclass[a4paper,12pt]{scrbook}

\usepackage{pdfsync} % Erlaubt das Hin- und Herspringen zwischen PDF und LaTeX-Quelltext

\usepackage[utf8]{inputenc} % Die hier verwendeten .tex-Dateien sind in UTF-8 kodiert
\usepackage{german} % Sprache ist Deutsch, wir verwenden alte Rechtschreibung
\usepackage{charter} % Unsere CI-Schrift ist Charter
\usepackage[protrusion=true,expansion=true]{microtype} % Typographische Feinheiten
\frenchspacing % schaltet vergrößerten Abstand nach Satzende aus

%Überschriftenformatierung
\setkomafont{sectioning}{\normalfont\bfseries} % Denselben Zeichensatz wie im Text auch für Überschriften verwenden
%\setkomafont{section}{\large}
%\setkomafont{subsection}{\normalsize}

% Einstellungen für URLs
\usepackage{url}
\urlstyle{same}
\renewcommand{\url}[1]{\textit{#1}}

% Listenformatierung
\usepackage{paralist} % ermöglicht \compactitem und \compactenum

% Schusterjungen und Hurenkinder verhindern
\clubpenalty = 10000
\widowpenalty = 10000

% Trennstellen beeinflussen
\usepackage[T1]{fontenc}
\hyphenation{} % besondere Trennstellen angeben, z. B. Hack-sport-gruppe, von Leerzeichen getrennt

% Erstzeileneinzug definieren
\setlength{\parindent}{1em} 

% Satzspiegel, Einzüge usw. definieren
\usepackage[top=2cm,bottom=3.5cm,left=2.8cm,right=3.6cm,headsep=6mm]{geometry}

% Frontmatter Römisch numeriert
\csname g@addto@macro\endcsname\frontmatter{\pagenumbering{Roman}}

%Leerseite einfügen
\newcommand{\leerseite}{\newpage\thispagestyle{empty}\mbox{}\newpage}

% Sorgt für Fußnoten *unter* Gleitobjekten
\usepackage{stfloats}
\fnbelowfloat

% Absatzabstände vereinheitlichen
\raggedbottom % schaltet flushedbottom ab

% umbrechbares schmale Leerzeichen definieren
\makeatletter % definiert \, und \,*, wovon letztes umbrochen werden kann
\DeclareRobustCommand{\,}{%
  \relax\ifmmode\mskip\thinmuskip\else\thinspace\fi
  \@ifstar{\hskip\z@skip}{}%
}
\makeatother

\usepackage{graphicx} % Graphikeinbindung
\usepackage{amsmath} % ermöglicht pmb = poor man's bold
\newcommand{\ultrafett}[1]{\pmb{\textbf{{#1}}}} % definiert ultrafett

\usepackage{ifthen} % Erlaubt einfache Programmierungen in LaTeX
\usepackage{calc} % Erlaubt Berechnungen

\begin{document} % Hier endet der Vorspann, und es beginnt der Dokumentinhalt

\frontmatter % Die ersten Seiten eines Buches, die hier mit römisch numeriert sind 

\vspace*{4cm} % 4 cm Abstand von oben, mit Stern, weil vorher keine Objekte auf der Seite sind

\thispagestyle{empty} % entfernt Kolumnentitel (= Titel in der Kopfzeile) und Seitenzahl
\fontsize{48}{60}\selectfont \ultrafett{Junghackerfibel} % Setzt die Schriftgröße auf 48 und den Zeilenabstand auf 50 

\vspace*{1cm}

\fontsize{24}{30}\selectfont \ultrafett{Wie lerne ich hacken?}

\vspace*{0.2cm}

\fontsize{24}{30}\selectfont \ultrafett{– für Kinder ab 12 Jahren} 
\normalsize % setzt die Schriftgröße auf normal zurück

\newpage

\vspace*{\fill} % Schiebt alles Nachfolgende bis zum nächsten Seitenumbruch unten auf die Seite, mit Sternchen auch dann, wenn die Seite eben erst angefangen hat.

\noindent Fassung vom \today.\\ % \\ = neue Zeile

\noindent Alle verwendete Firmen-, Markennamen und Warenzeichen sind Eigentum der jeweiligen Inhaber. (Wer hätte das gedacht?)\\ 

\noindent Die Verwendung der beschriebenen Vorschläge und Anleitungen erfolgt auf eigene Gefahr. 

%!TEX root = Junghackerfibel_Buch.tex
%!TEX encoding = UTF-8 Unicode

\chapter{Vorwort}

Dieses Buch 
 % Vorwort einbinden; Dateiendung hier nur zur Verständlichkeit

\tableofcontents % Gibt das Inhaltsverzeichnis aus.

\mainmatter

%!TEX root = Junghackerfibel_Buch.tex
%!TEX encoding = UTF-8 Unicode

\chapter{Einführung}

\section{Was ist Hacken?}

Hacken ist eine Denkweise und eine Lebenseinstellung. Denkfaule Menschen fragen: Wofür muß ich das wissen? Hacker fragen: Wofür könnte ich das wohl verwenden? Hacken ist eine Denkweise, die das gesamte Leben durchzieht. Die Denkweise des Hackens läßt sich ebenso auf Kochen wie auf Sprache oder soziale Interaktion anwenden. 

Die ersten Hacker waren wohl diejenigen Menschen, die zwei Steine aufeinandergeschlagen und festgestellt haben, daß dabei scharfkantige Teile abplatzen können, oder wie man mit einem Stein und einem Stock Feuer machen kann. Hacker erblicken weitere wichtige Ahnen in beispielsweise dem Komponisten Johann Sebastian Bach, dem Dichter William Shakespeare oder dem Aufklärer und Verleger Joachim Heinrich Campe, der die deutsche Sprache vielleicht in ähnlich gewaltiger Weise bereichert hat wie Shakespeare die englische. 

Hacken ist also keineswegs auf Rechner und Netzwerke beschränkt, nicht einmal auf Technik, sondern bezieht sich auf Systeme aller Art. 

Dieses Buch befaßt sich mit einem umfassenden Einstieg ins Hacken, denn es ist festzustellen, daß viele Hacker \emph{nicht} damit angefangen haben, an Computern und Programmen herumzubasteln. Für viele war der Einstieg ein ganz anderer: Sprache. Für wieder andere begann es mit Legosteinen, ging über Holzbearbeitung zu Metallbearbeitung zu Elektronik. Beide Beispiele haben eines gemeinsam: eine niedrige Einstiegsschwelle. Der Ansatz dieses Buches will den Einstieg möglichst niedrig gestalten. Alles andere ergibt sich dann von alleine.

Wenn ein junger Mensch fragt: \emph{Wie werde ich Hacker?} werden viele ihn auf bestimmte Texte verweisen, welche die Hürde so hoch legen, daß sie als Einstieg illusorisch ist. Junge Menschen vom Hacken abzuhalten ist arrogant und widerspricht dem aufklärerischen und philantropischen Anspruch, den ein Hacker eigentlich haben sollte. Ich werde deshalb in diesem Buch auch immer wieder auf die ethischen Aspekte des Hackens eingehen.  % Kapitel Einführung einbinden

%!TEX root = Junghackerfibel_Buch.tex
%!TEX encoding = UTF-8 Unicode

\chapter{Voraussetzungen}

Dieses Buch ist geschrieben für Kinder ab 12 Jahren. Jugendliche und Erwachsene werden es aber ganz bestimmt genauso mit Gewinn lesen. 

Die Anleitungen und Beispiele in diesem Buch beziehen sich auf den Apple Macintosh, und zwar aus mehreren Gründen:

\begin{compactitem}[–]
\item Macintosh-Rechner sind ideale Rechner für Kinder, weil man sie ziemlich angstfrei verwenden kann, man sich keine größeren Sorgen um Viren und andere Schadprogramme machen muß, und sie eine niedrige Einstiegsschwelle in den Umgang mit Rechnern generell bieten.
\item Macintosh-Rechner sind in der Hacker-Szene mit Abstand die beliebtesten Geräte, denn man kann damit alles machen: Man hat Zugriff auf Macintosh-Programme, wenn man auch mal etwas Normales arbeiten muß, mehr oder weniger sämtliche Unix-\emph{Software}, kann mehr oder weniger sämtliche Betriebssysteme in einer Box laufen lassen und bekommt keine TPM-Komponente aufs Auge gedrückt, die man nicht haben will (TPM = \emph{Trusted Platform Module}), und das Ganze sieht überhaupt nicht zum Kotzen aus. Abgesehen davon sind die Geräte in der Regel schnell und langlebig, dafür verlangt Apple Premium-Preise, welche die Leute, die damit arbeiten, offenbar für ihr Arbeitsgerät zu zahlen bereit sind. Ein Schreiner kauft sich für sein Handwerker auch die beste Säge und den besten Hobel, und Kopfarbeiter sollten auch das beste Werkzeug wählen.
\item Menschen, die mit dem Macintosh großgeworden sind, kommen mehr oder weniger mit jedem anderen Rechner zurecht, wohingegen Leute, die mit Linux oder Windows großgeworden sind, nie lernen, daß Dinge auch einfach sein können. Der Sinn für Ästhetik und Einfachheit – nicht zuletzt die Ästhetik der Einfachheit – wird man mit einem Macintosh als Arbeitsgerät vermutlich so verinnerlichen, daß sich eine Aversion gegen unnötige, lieblose und denkfaule Kompliziertheit entwickelt – und das ist gut!
\end{compactitem}

\noindent Man kann also vom blutigen Anfänger bis zum Guru mit demselben Gerät arbeiten. Ich will hier weder Werbung machen noch große Diskussionen vom Zaun brechen. Deshalb: Genug davon!

Mit Windows kommt man als Hacker nicht weit, mehr gibt es dazu nicht zu sagen. Wer sein Windows vom Rechner zu werfen bereit ist oder vor der Anschaffung eines Rechners steht, möge sich Linux – Ubuntu, Kubuntu, Mint – oder gleich PC-BSD anschauen. Irgendwie wird es auch möglich sein, Windows und Linux bzw. PC-BSD zu einer friedlichen Koexistenz auf demselben Rechner überreden lassen, aber ich werde nie erfahren, wie das geht. Viele der Kommandozeilenbefehle werden unter Unix-artigen Betriebssystemen genauso oder ähnlich funktionieren.

Es muß nicht der neueste Macintosh sein. Für die jüngsten Leser reicht ein alter Macintosh, er sollte aber bereits einen Intel-Prozessor besitzen, und es sollte zumindest Mac OS X 10.6 \emph{Snow Leopard} darauf laufen: So ein Gerät bekommt man mitunter geschenkt oder günstig im Internet. Wenn Du größer wirst, wirst Du irgendwie schon einen neueren Macintosh bekommen, und der alte Macintosh wird dann Dein Bastelgerät, auf dem Du FreeBSD installierst und Deinen ersten eigenen Server aufsetzt. % Kapitel Voraussetzungen einbinden

%!TEX root = Junghackerfibel_Buch.tex
%!TEX encoding = UTF-8 Unicode

\chapter{Lies und finde Spaß daran}

Gewöhne Dir an, regelmäßig zu lesen, zum Beispiel eine halbe Stunde vor dem Einschlafen oder eine halbe Stunde nach dem Erwachen – vielleicht zumindest am Wochenende. Je mehr Du liest, desto leichter wird es Dir fallen. Laß Dich nicht von der Dicke von Büchern abschrecken: Lies Zeile für Zeile, Seite für Seite – und irgendwann wirst Du feststellen, daß Du ein ganzes Buch gelesen hast. 

Frag andere Leute, welche Bücher sie gelesen haben und empfehlen können, Deine Eltern, Deine Freunde, Deine Lehrer. Es gibt Bücher, die sind der helle Wahnsinn. Finde diese Bücher! Menschen in Deinem Alter lesen zum Beispiel gerne \emph{Die drei ???}, \emph{Asterix}, \emph{Perry Rhodan}, Karl May oder auch die \emph{Lustigen Taschenbücher} – und natürlich \emph{Harry Potter}: \emph{Harry Potter} ist eine besondere Empfehlung, denn es ist vordergründig eine spannende Geschichte, aber sie ist doppelbödig, vielschichtig und tiefsinnig, also genau das, was gute Literatur ausmacht.

Frag, ob Du die Pause in der Schulbibliothek verbringen darfst: Wenn sowieso eine Aufsichtsperson da ist, gibt es keinen Grund, weswegen die Schule Dir das verweigern sollte. Vielleicht kannst Du Deine Freunde davon überzeugen, die Schulbibliothek als Euer Reich zu übernehmen. Du willst in der Pause lieber lernen als die Zeit auf dem Pausenhof zu vertrödeln, und die Schule hindert Dich daran? Das ist ja unglaublich! Bitte die Schülervertretung oder den Elternbeirat um Unterstützung.

In der Schulbibliothek könntest Du Dir zuerst einmal das Ordnungssystem, die Signaturen auf dem Buchrücken und den Katalog erklären lassen – oder Dir selbst erklären. Es ist völlig egal, was Du liest: Du kannst einen Asterix nach dem anderen lesen oder den Brockhaus in einem Band von A bis Z durchlesen oder mit geschlossenen Augen zufällig einen Artikel auswählen. Friß Dir Wissen an, je mehr Wissen Du Dir angefressen hast, desto leichter wirst Du Dir weitere Sachen merken können. Geh danach, was Dich interessiert: Wenn es Fußball ist, dann lies alles über Fußball!

Besorg Dir das Buch \emph{Bildung} von Dietrich Schwanitz – damit wirst Du es in allen nicht-naturwissenschaftlichen Fächern leichter haben, und Du hast mehr Zeit für Deine Hacksportaktivitäten. % Kapitel Lesen einbinden

%!TEX root = Junghackerfibel_Buch.tex
%!TEX encoding = UTF-8 Unicode

\chapter{Erlerne das Zehnfingersystem}

Du mußt tippen lernen – mit Zehnfingersystem – ohne auf die Tasten zu schauen: Sobald Du einen Computer hast – gerne auch mit einer Schreibmaschine – solltest Du lernen, mit Zehnfingersystem zu tippen. Bring es so früh wie möglich hinter Dich.  Mach einen Schreibmaschinenkurs an der Volkshochschule oder such Dir einen Kurs im Internet. Fang gar nicht erst an mit dem sog. Adler-such-System (»spähen und zuschlagen«), denn sonst sind Deiner Tippgeschwindigkeit Grenzen gesetzt, und Du gibst eine traurige Figur ab. Lerne es gleich richtig. Irgendwann fließen Deine Gedanken unter Umgehung des Sprachzentrums direkt in die Tastatur, und Du tippst schneller als Dein Schatten. % Kapitel Tippen einbinden

%!TEX root = Junghackerfibel_Buch.tex
%!TEX encoding = UTF-8 Unicode

\chapter{Verwende einen Texteditor}

Ein Texteditor ist ein Programm, das darauf spezialisiert ist, Textdateien zu bearbeiten. Dieses Programm ist für einen Hacker das zentrale Werkzeug. Auf dem Macintosh empfiehlt sich zum Einstieg \emph{Textwrangler} – kostenlos zu beziehen von \emph{http://www.barebones.com}.

Ein Texteditor ist etwas anderes als ein Textverarbeitungsprogramm: Um das zu illustrieren öffnen wir zum Einstieg mit \emph{Textwrangler} eine \emph{Word}-Datei mit \emph{.doc}-Dateiendung: Dieser Salat enthält den eigentlichen Text und die Formatierungsanweisungen. % Kapitel Texteditor einbinden

%!TEX root = Junghackerfibel_Buch.tex
%!TEX encoding = UTF-8 Unicode

\chapter{Lerne die Kommandozeile kennen}

\section{Terminal-Fenster und Kommandozeile}

Starte das Programm \emph{Terminal}; es öffnet sich ein Fenster – ein Terminal-Fenster. Das Wort \emph{bash} in der Titelzeile des Fensters steht für \emph{Bourne again shell}. Die \emph{bash} ist eine Weiterentwicklung der \emph{Bourne shell}, kurz \emph{sh}. Als \emph{Shell} (engl. Schale) bezeichnet man die Benutzeroberfläche von Unix-artigen Betriebssystemen. Die Shell zeigt zunächst nur einen leeren Textbildschirm mit einer Kommandozeile, die etwa so aussieht:

\begin{verbatim}
xerxes:~ nescio$
\end{verbatim}

\noindent Am Anfang steht der Name des Rechners, auf dem man sich gerade befindet, nach dem Doppelpunkt folgt der Name des Ordners, in welchem man sich gerade befindet: Die Tilde ist die Kurzform für \emph{home} – den eigenen Ordner des Benutzers, der einem auch als ›Heimatverzeichnis‹ begegnet. Da Unix-artige Betriebssysteme dafür gemacht sind, daß sich mehrere Menschen einen Rechner teilen, bekommt jeder Benutzer seinen eigenen Ordner, in dem er mehr oder weniger machen kann, was er will. 

Nach dem Leerzeichen folgt der Name des Benutzers, der man gerade ist. Am Ende folgen ein Dollarzeichen und ein Leerzeichen. Das Dollarzeichen bedeutet, dass man normaler Benutzer ist. Es gibt noch den privilegierten Benutzer \emph{root}, der hat an dieser Stelle eine Raute – die bekommen wir aber wahrscheinlich nie zu Gesicht, zumindest nicht auf dem eigenen Rechner.

\section{Navigation im Ordnersystem}

Wie findet man heraus, in welchem Ordner man sich gerade befindet? Tippe in die Kommandozeile ein:

\begin{verbatim}
pwd
\end{verbatim}

\noindent und betätige die Absatztaste bzw. Zeilenschaltung (\emph{Return}). Unser home-Ordner befindet sich also hier:

\begin{verbatim}
/Users/nescio/
\end{verbatim}

\noindent Als nächstes schauen wir uns hier mal genauer um. Probiere nacheinander mal folgende Kommandos aus:

\begin{verbatim}
ls
ls -a
ls -l
ls -l -a
ls -la
tree -L 2
\end{verbatim}

\noindent Der letzte Befehl dürfte jetzt noch nicht funktionieren – wir probieren ihn später nochmal aus.

Was haben wir gerade getan? Wir haben Programme gestartet und den Programmen unterschiedliche Befehle mit auf den Weg gegeben. Das erste Programm \emph{ls} steht für \emph{list} und listet auf, welche Dateien sich in einem Ordner befinden.

\begin{verbatim}
.
..				
Desktop
Documents
Downloads
Library
Movies
Music
Pictures
Public
\end{verbatim}

\noindent Es fällt auf, daß uns der Finder an denselben Stellen manche Ordner mit deutschem Namen anzeigt, der Ordner \emph{Schreibtisch} heißt in Wirklichkeit \emph{Desktop}. Man kann sich durch diese Ordner hangeln, wie man es im Finder gewohnt ist, allerdings gibt es dafür Kommandozeilenbefehle:

\begin{verbatim}
cd Documents
cd .
cd ..
\end{verbatim}

\noindent Mit dem ersten Befehl wechseln wir in den Ordner \emph{Documents}, mit dem zweiten Befehl wechseln wir in den aktuellen Ordner (Da sind wir bereits, es passiert also im Grunde gar nichts.), mit dem dritten Befehl wechseln wir in den übergeordneten Ordner.

Wenn wir dem Kommando \emph{ls} den Parameter \emph{-a} mitgeben, dann listet uns das Programm alles auf, was sich im aktuellen Ordner befindet. Manche Dateien werden unsichtbar gemacht, indem man den Dateinamen mit einem Punkt beginnen läßt. Der Parameter \emph{-l} steht für \emph{long}, und die Ausgabe für eine einzelne Zeile sieht etwa so aus:

\begin{verbatim}
drwxr-xr-x+  546 nescio  staff   18564  9 Jun 14:20 Documents
\end{verbatim}

\noindent Was die Angaben alle bedeuten, werden wir später erfahren.

\section{Homebrew und weitere Programme installieren}

Wir brauchen  zuerst einmal \emph{Homebrew} von \emph{http://brew.sh}. Die Installationsanweisung steht auf der Homebrew-Seite und lautet – jedoch ohne Zeilenumbruch – etwa so:

\begin{verbatim}
ruby -e "$(curl -fsSL https://raw.githubusercontent.com/Homebrew
/install/master/install)"
\end{verbatim}

\noindent Wir lassen die Installation ablaufen und sind danach bereit, weitere Kommandozeilenprogramme zu installieren. 
Typische Programme für den Anfang wären \emph{wget}, \emph{youtube-dl}, \emph{rtmpdump}, \emph{ffmpeg} und \emph{tree} – und das geht mit:

\begin{verbatim}
brew install wget
brew install youtube-dl
brew install rtmpdump
brew install ffmpeg
brew install tree
\end{verbatim}

 % Kapitel Kommandozeile einbinden



\backmatter

% \listoffigures % Abbildungsverzeichnis – haben wir derzeit noch nicht

% \listoftables % Tabellenverzeichnis – haben wir derzeit noch nicht

\end{document}
